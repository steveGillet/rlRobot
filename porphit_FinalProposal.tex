\documentclass[conference]{IEEEtran}
% \IEEEoverridecommandlockouts
% The preceding line is only needed to identify funding in the first footnote. If that is unneeded, please comment it out.
\usepackage{cite}
\usepackage{amsmath,amssymb,amsfonts}
\usepackage{algorithmic}
\usepackage{graphicx}
\usepackage{textcomp}
\usepackage{xcolor}
\def\BibTeX{{\rm B\kern-.05em{\sc i\kern-.025em b}\kern-.08em
    T\kern-.1667em\lower.7ex\hbox{E}\kern-.125emX}}

\begin{document}

\title{PorphIt: PPO Robot Manipulator Morphology Design}

\author{\IEEEauthorblockN{Steve Gillet}
\IEEEauthorblockA{\textit{University of Colorado Boulder Robotics} \\
\textit{Email: steve.gillet@colorado.edu}}
% Add more authors if team members are known, e.g.:
\and
\IEEEauthorblockN{Jay Vakil}
\IEEEauthorblockA{\textit{Email: Jay.Vakil@colorado.edu}}
% \and
% \IEEEauthorblockN{Team Member 3}
% \IEEEauthorblockA{\textit{Email: member3.email@example.com}}
}

\maketitle

\begin{abstract}
High costs, lack of tailorability, and lack of retrofitting are some of the most cited problems for the adoption of the robotics in industrial settings \cite{mckinsey2019industrial}.
We want to address this problem by creating an automation pipeline through which a use case is identified and a robot optimized for that use case is created.
By matching the robot to the task we can ensure that robot can perform that task and only the parts needed for that task are used, reducing cost.
Here we describe the process by which robot manipulator morphologies are optimized to particular tasks using a PPO agent.
\end{abstract}

\begin{IEEEkeywords}
reinforcement learning, robot morphology, manipulator design
\end{IEEEkeywords}

\section{Introduction and Motivation}
% Content here

\section{Background and Related Work}
% Content here

\section{Methods}
% Content here

\section{Experimental Setup}
% Content here

\section{Timeline and Milestones}
% Content here

\section{Risks and Mitigations}
% Content here

\section{Resources}
% Content here

\section{Ethics and Safety}
% Content here

\section{Evaluation Plan}
% Content here

\section{Expected Results}
% Content here

\section{Contributions and Roles}
% Content here

\bibliographystyle{IEEEtran}
\bibliography{references} % Assuming a references.bib file

\end{document}